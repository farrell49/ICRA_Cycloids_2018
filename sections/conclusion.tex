The aim of this work was to determine the in-use characteristics of a cycloidal drive designed for a robotic application through an extended break-in test and efficiency testing.
Through the work, this study demonstrates a cycloidal actuator with a ratio of 59:1 and three phased cycloid disks that achieves a maximum efficiency of 80\% and does not show a constant efficiency through its torque profile as suggested by previous sources.
This research shows that these drives efficiencies behave very similarly to other typical reduction drives for similar applications like Harmonic Drives and Planetary gears.
This actuator compares closely to its Harmonic Drive counterpart in efficiency performance.
If backlash is acceptable in the system, a cycloidal drive has the distinct advantage of being customized into the housing using simple manufacturing techniques allowing tighter integration into a robotic system as well as a potential 2x specific torque gain.
While the efficiency and load capacity may be higher when using rolling elements for the input and output pins, this research shows that efficiencies in a similar range to other high reduction drives can be achieved when these heavy and large rolling elements are not used.
An item of interest that has not been characterized by the community is the lifetime characteristics of this cycloid design style and this is left by the authors for future work.
Cycloidal drives of this design style are quite comparable to similar use-case drives and should be considered in high reduction applications.

