
The long duration testing was performed first on the actuator.
Because the motor was designed for short bursts of motion, these tests were done at lower torques to prevent the motor from overheating to allow extended duration testing.
The total test time prior to these duty cycle tests was approximately 5.2 hours to bring up and check out the actuator testbed system.
Once this checkout was complete, the 100 hours of duty cycle testing were conducted over the course of 11 days with the drive cycle presented in Table \ref{table_2}.
Three of the torque/speed combinations in forward and reverse are plotted on Fig \ref{long_run} to show the general characteristic trends seen in actuator performance.

After the long duration testing was complete, it is evident that the actuator had sufficiently broken-in and had achieved steady state performance as evident by Fig \ref{long_run}. At this time the pure efficiency cycles were run.
A profile of speeds and torques (see Fig \ref{eff_profile}) were run on the actuator to show the relationship between speed, torque, and efficiency.
This profile was run three times and the results at each torque and speed combination were averaged (see Fig \ref{eff_results}).

