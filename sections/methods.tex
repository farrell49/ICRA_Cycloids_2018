\begin{figure}[!b]
   \centering
   \includegraphics[width=0.75\linewidth]{images/test_stand}
   \caption{Experimental Test Setup.
   The cycloid actuator is mounted to structure via the load cell.
   There is a speed increase so the brake can generate enough torque on the system.
   Not pictured is the controlling computer, motor driver, and high voltage supply.}
   \label{test_setup}
\end{figure}

The intent of testing the cycloidal drive is to experimentally determine and compare the in-use efficiency results to the published performance data for a comparable harmonic drive.
To accomplish this, the actuator was mounted to a Futek TF600 5000inlb load cell to measure output torque of the actuator.
The load cell signal was collected through a analog to digital converter and converted to standard units on the motor driver.
A verification of torque readings was completed using a calibrated torque wrench to ensure accuracy of the conversion.
Load was regulated by a Magtrol HB-1750 hysteresis brake.
A 36:1 speed increase was added via three chain stages between the output of the actuator and the hysteresis brake to achieve the desired applied loads.
The hysteresis brake was powered using a separate 24V Lambda-TDK power supply that was controlled through a RS-485 communication link to the test computer.
NASA's 'turbodriver' motor driver was used for commanding motor currents.
The motor driver was powered by a TDK-Lambda 12V supply for logic power and a TDK-Lambda 150V and 5A supply for high voltage power.
A hysteresis current controller was used on the motor drive to accurately provide torque producing current to the motor.
The motor driver monitored the actuator power by measuring torque from the load cell and speed with an incremental encoder located at the actuator motor shaft.
The test computer monitored the high voltage supply and recorded voltage and current to determine input power to the system and received data from the motor driver to calculate output power.
The test setup is shown in Fig \ref{test_setup}.

Due to the tightly integrated actuator design, the motor and cycloid cannot be separated to purely isolate the losses in the cycloid.
The efficiency map of the motor over its torque and speed range was provided by Parker Motors.
For calculation purposes, this table is used as a lookup table for efficiency of the motor given the current motor velocity and rms input current.
While this does generate a level of uncertainty in the data, these motors are mass manufactured and defects are assumed to be small.
A different Parker motor was analyzed and compared to the data provided and its performance matched within 5\% of the provided data. Therefore, the error in the motor efficiency map for this test setup is assumed to be small and would not heavily influence the perceived trends in the results.
Power losses in the motor driver were also taken into consideration by calculating power losses in the IGBT that drives the motor.
Instantaneous current draw and a switching frequency of 12 kHz were used for the calculation, neglecting small leakage losses \cite{IGBTPower}.

\begin{table}[t]
	\vskip0.2cm
	\caption{Long Run Drive Cycle}
	\label{table_2}
	\begin{center}
		\vskip-0.2cm
		\begin{tabular}{|c||c||c|}
			\hline
			Time (s) & Velocity (rad/s) & Torque (Nm)\\
			\hline
			150 & 1.0 & 0.0\\
			\hline
			150 & -1.0 & 0.0\\
			\hline
			60 & 0.5 & 26.0\\
			\hline
			60 & -0.5 & 26.0\\
			\hline
			150 & 1.5 & 10.0\\
			\hline
			150 & -1.5 & 10.0\\
			\hline
			30 & 1.0 & 50.0\\
			\hline
			30 & -1.0 & 50.0\\
			\hline
			300 & 0.5 & 18.0\\
			\hline
			300 & -0.5 & 18.0\\
			\hline
		\end{tabular}
	\end{center}
\end{table}

\begin{figure}[!b]
   \centering
   \includegraphics[width=\linewidth]{images/eff_test_profile_v4}
   \caption{Testing profile for efficiency.
   At each speed step, torque is ramped up through five different levels, then the speed is increased.
   At the last step, the maximum of the supply was reached so motor velocity dropped.}
   \label{eff_profile}
\end{figure}

The system was tested in two separate ways, an efficiency cycle and a long term drive cycle.
The efficiency cycle test was run after the long term drive cycle to ensure steady state performance before cycling through a set of velocities and torques.
The actuator is subjected to eight velocity steps increasing 0.25 rad/s each time.
In each velocity step, the torque is ramped up and maintained for 15 seconds at values of 1Nm, 15Nm, 52Nm, 94Nm, and 189Nm.
This testing profile can be seen in Fig \ref{eff_profile}.
The long term drive cycle was run continuously each day for 6 to 12 hours with the drive cycle shown in Table \ref{table_2}.
The total runtime of the system, not including the initial checkout and verification of the actuator, has been over 300 hours.


\begin{figure*}[!t]
	\centering
	\includegraphics[width=0.9\linewidth]{images/total_runtime}
	\caption{Efficiency over time for three different speed/torque profiles during the drive cycle.
		The forward motion can be seen with the dotted line, reverse with the solid line.
		At the onset of testing, visible efficiency gains are made.
		As each day begins, there is a clear warm-up period before steady state.
	}
	\label{long_run}
\end{figure*}


\begin{figure*}[t]
	\centering
	\includegraphics[width=0.8\linewidth]{images/eff_final}
	\caption{Grouping of average efficiencies at each torque step.
		Efficiency depends heavily on torque, and slightly on speed.}
	\label{eff_results}
\end{figure*}

It should be noted that the actuator was used briefly in the robot validation after initial development and construction of the prototype wheel module.
The total time of use was approximately three hours.
Afterwards, it was removed from the wheel module and subjected to the characterization that is discussed in this work.
The motor has a continuous current rating of 4.3 A\textsubscript{rms} and a peak rating of 15A\textsubscript{rms}.
The actuator was designed to be liquid cooled to allow operations above the continuous values, but due to the test equipment available, the liquid cooling could not be achieved during testing.
The actuator was designed for short bursts of motion rather than continuous cycling, therefore, lower torques were used during the long duration testing to avoid thermal issues and allow continuous motion to determine the trends in performance over time.
The motor was tested to approximately 6A\textsubscript{rms} during efficiency testing due to the limit of the power supply.
Also, the motor driver's rated limits are 150V, therefore the actuator's maximum rated speeds could not be tested.
The nominal cycle of the actuator as seen in Table \ref{duty_cycle} is still achievable and has been tested.

The long duration testing was first performed on the actuator for 100 hours. Because the motor was designed for short bursts of motion, these tests were done at lower torques to prevent the motor from overheating to allow extended duration testing. The total test time prior to these duty cycle tests was approximately 5.2 hours to bring up and check out the actuator testbed system.
After the first 100 hours, it was evident that the actuator had sufficiently broken-in and had achieved steady state performance as evident by Fig \ref{long_run}. At this point, the first pure efficiency cycles were run to understand performance. A profile of speeds and torques (see Fig \ref{eff_profile}) were run on the actuator to show the relationship between speed, torque, and efficiency. Afterwards, the system was deconstructioned to again validate the load cell readings and reconstructed.
After this validation, the actuator was run for 200 additional hours using the long duration test cycle. Then, a final efficiency cycle was run in the same way as before. 

