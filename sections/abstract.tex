Currently, harmonic drives are the primary speed reducer for robotic applications where a high reduction in a small package is required. 
Cycloidal drives are an alternative option for high reduction, small package, use-cases with the advantage of a higher specific torque and the ability to customize and integrate the drive for the application.
These compact style cycloidal drives have been well studied in theory and simulation for their performance, but very little data is available on their actual performance over time. 
This paper presents experimental data on performance of a cycloidal drive designed for a Lunar or Martian rover application. 
Burn-in time efficiency curves and torque/speed efficiency profiles are computed after running the drive through 129k output cycles (7.6M input cycles) over the course of over 300 hours of testing. 
The study finds that substantial burn-in time may be required for steady-state performance, but peak efficiencies of 81\% can be achieved. 
Also, the efficiency is shown to be dependent on the torque through the actuator.
This work demonstrates a customized cycloidal drive in a space application that is comparable to a harmonic drive in efficiency performance, with a 2x increase in specific torque, suggesting the application of cycloidal drives could grow tremendously in robotic designs. 
