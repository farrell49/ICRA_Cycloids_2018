Currently, harmonic drives are the go-to speed reducer for robotic applications where a high reduction in a small package is required. However, cycloidal drives can also fit this mold with the ability to customize a high reduction drive that can carry high torque in a small package. These compact style cycloidal drives have been well studied in the theory and simulation for their performance, but very little data is available on their actual performance over time. This study used a cycloidal drive designed for a robotic application and ran it through 37K (TODO) cycles over 80 (TODO) hours of testing to determine burn-in time, efficiency curves, and efficiency profiles over time to determine its comparison to a Harmonic Drive in-use. The study finds that substantial burn-in time may be required for steady-state performance, but peak efficiencies of 77\% can be achieved. Also, the efficiency is dependant on the torque through the actuator, contrary to multiple previous studies. This work demonstrates a cycloidal drive in a robotic application that is comparable to a Harmonic Drive, suggesting the application of cycloidal drives could grow tremendously in robotic designs. 